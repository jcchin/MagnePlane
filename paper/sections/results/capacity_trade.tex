The original Hyperloop proposal suggested
that the pod would likely be able to carry about 28 passengers \cite{Musk}.
To meet the market demand, the frequency at which pods depart could be
increased or decreased as necessary. A high pod frequency could be
problematic as it would require a large number of pods to be maintained at
the end points and may not provide enough time for passengers to board comfortably.
Thus, it is of particular interest to examine how the performance of the system
is affected over a range of pod capacities. Increasing the capacity would allow
the operator to have fewer pods taking off at a lower frequency to meet
the market demanded passenger throughput. The overall benefits of
changes in pod capacity can be more accurately determined by analyzing the
sensitivity of energy consumption and operating cost to pod capacity.
In this analysis, the number of passengers per pod is varied from 10 to 100.
At each quantity, the estimated recurring energy cost and cost per capita are recorded.
% \begin{figure}
% 	\centering
% 	\includegraphics[width=.45\textwidth]{../../images/graphs/capacity_trades/passengers_vs_energy.png}
% 	\caption{Yearly Energy Cost vs. Passengers per Pod}
% 	\label{fig:energy_cost_vs_passengers}
% \end{figure}
\begin{table}
    \begin{center}
        \begin{tabular}{| l | l | l | l | l | l | l | l | l | l | l |}
        \hline
        Passengers Per Pod & 10 & 20 & 30 & 40 & 50 & 60 & 70 & 80 & 90 & 100 \\ \hline
        Energy Cost (\$M) & 28.2 & 28.6 & 29.0 & 29.5 & 29.9 & 30.3 & 30.7 & 31.1 & 31.6 & 32.0 \\ \hline
        Total \$/Passenger & 226 & 113 & 76 & 57 & 45 & 38 & 33 & 29 & 25 & 23 \\ \hline
        \end{tabular}
        \caption{Energy cost for various pod capacities}
        \label{tab:energy_cost_vs_passengers}
    \end{center}
\end{table}

\Cref{tab:energy_cost_vs_passengers} shows the relationship between yearly
energy consumption and the number of passengers per pod produced by the system model.
It is shown that, for the given operating condition, an order of magnitude
increase in pod capacity only results in a 15\% increase in yearly energy consumption.
This, in conjunction with the previously discussed structural analysis,
indicates that the cost associated with changing pod capacity is small.
This relationship is significant because it means that the Hyperloop operator
can specifically set the pod capacity to whatever value is necessary to meet a
particular market demand, without costly changes in performance or design.
Furthermore, this makes it possible for future researchers to consider making
Hyperloop pods modular. It is possible that, instead of having one large pod
carrying a fixed number of passengers, the operator could have multiple pods
that carry a small number of passengers. Each of these pods would link together until the capacity
of each individual flight matched demand.
This would allow the Hyperloop to handle high densities of passengers during
peak travel times without having to increase pod frequency to prohibitive levels.
Then, during lighter travel times, the operator could link fewer pods together
to reduce the gross weight of each flight in order to reduce unnecessary energy consumption.
The cost per passenger also rapidly declines, before tapering off at capacities
not suited for the tube pressure chosen for this particular parameter sweep.
