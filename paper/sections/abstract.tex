\begin{abstract}
	MagnePlane is a Hyperloop-derivative concept proposed as a faster,
	cheaper alternative to high speed rail and traditional short-haul aircraft.
	It consists of a passenger pod traveling through a tube under light vacuum and
	propelled and levitated by a combination of permanent and electro-magnets.
	The concept is motivated by NASA's research thrusts driven by a growth in
	demand, sustainability, and technology convergence for high-speed transport.
	Magneplane is a radical departure from other advanced aviation concepts at
	first glance, however it remains an aeronautics concept tackling the exact
	same strategic golas of low-carbon propulsion and ultra-efficient vehicles.

	The feasibility of this novel concept is investigated through a technical
	and cost perspective, focusing on vehicle aero and thermodynamics, structures,
	electromagnetics, weight and mission.
	A high-level sizing study is performed to determine total system costs and
	power usage given a parametric design varying tube area,
	pressure, pod speed, and passenger capacity.
	A vehicle sizing method is developed, using open-source toolsets focusing on
	the strong coupling between the two largest systems: the tube and the pod.
	The airborne vehicle requires many of the same technologies and expertise
	under development for next-generation aircraft and its high public visibility
	and rapid development make it an ideal candidate as an aircraft
	technology driver and test bed.
\end{abstract}
