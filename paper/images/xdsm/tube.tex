\documentclass{article}
\usepackage{geometry}
\usepackage{amsfonts}
\usepackage{amsmath}
\usepackage{amssymb}
\usepackage{tikz}

\input{diagram_border}

\begin{document}

\input{diagram_styles}

\begin{tikzpicture}

  \matrix[MatrixSetup]
  {
    %Row 1
    \node [Function] (SubTube) {\Large \TwolineComponent{6em}{Submerged}{Tube}}; &
    &
    &
    &
    &
    &
    \\
    &
    \node [AnalysisGroup] (Temp) {\Large \TwolineComponent{6em}{Tube}{Temperature}}; &
    \node [DataInter] (PropMech-Temp) {$T_\text{tube}$}; &
    \node [DataInter] (SSVacuum-Temp) {$T_\text{tube}$}; &
    &
    \node [DataInter] (Power-Temp) {$T_\text{tube}$}; &
    \\
    &
    &
    \node [Function] (PropMech) {\Large \TwolineComponent{6em}{Propulsion}{Mechanics}}; &
    &
    &
    \node [DataInter] (Power-PropMech) {$Pwr_\text{required}$}; &
    \\
    &
    &
    &
    \node [FunctionGroup] (SSVacuum) {\Large \TwolineComponent{6em}{Steady State}{Vacuum}}; &
    &
    &
    \\
    &
    &
    &
    &
    \node [Function] (Vacuum) {\Large Vacuum}; &
    \node [DataInter] (Power-Vacuum) {$E_\text{total}, Pwr_\text{total}$}; &
    \node [DataInter] (TubePylon-Vacuum) {$w_\text{total}$}; \\
    %Row 6
    &
    &
    &
    &
    &
    \node [Function] (Power) {\Large Tube Power}; &
    \\
    &
    &
    &
    &
    &
    &
    \node [Function] (TubePylon) {\Large \TwolineComponent{6em}{Tube}{and Pylon}}; \\
    %Row 8
  };

  \begin{pgfonlayer}{data}
    \path
    % Horizontal edges
    (Temp) edge [DataLine] (Power-Temp)
    (PropMech) edge [DataLine] (Power-PropMech)
    (Vacuum) edge [DataLine] (TubePylon-Vacuum)
    % Vertical edges
    (PropMech-Temp) edge [DataLine] (PropMech)
    (SSVacuum-Temp) edge [DataLine] (SSVacuum)
    (Power-Temp) edge [DataLine] (Power)
    (TubePylon-Vacuum) edge [DataLine] (TubePylon)
    ;
  \end{pgfonlayer}

\end{tikzpicture}

\end{document}
